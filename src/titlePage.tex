\thispagestyle{empty} %La page de titre n'a pas de numéro
\setcounter{page}{0} %On remet donc le compteur à 0

\begin{figure}[H]
\centering
\includegraphics[width=5cm]{./extras/logos/logo.pdf}
\end{figure}

\vspace{3cm}

\begin{center}

{\color{orange}\rule{\linewidth}{0.8mm}}
\vspace*{0mm}

\Huge{\textbf{TP1: Filtres passifs à éléments
inductifs et capacitifs localisés}}
{\color{orange}\rule{\linewidth}{0.8mm}}

\vspace{0.5cm}
\Large{\theauthor} \\
\small{(Groupe TP4)}\\
\Large{\today}
\end{center}
 
\vspace{3cm}

\begin{myabstract}
Au cours de ce TP de filtrage analogique, nous allons répondre à un cahier des charges de trois fonctions de filtrages différentes. 
Le but est de confronter leur conception théorique vue dans la préparation (gabarit et circuits) avec leurs performances réelles puis simulées.
Cela va nous permettre de juger leur efficacité et leur fiabilité grâce à la méthode de conception avec les tables de coefficients $g_i$, tout cela afin de vérifier si le cahier des charges a bien été respecté.
\end{myabstract}

\clearpage
